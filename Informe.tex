\documentclass{udpreport}
\title{Comprobación del funcionamiento del algoritmo STP e implementación de VLAN}
\author{Integrantes: Thomas Muñoz, Ignacio Yanjari, Dagoberto Navarrete, Ignacio López.}
\date{19 de Mayo de 2016}
\usepackage{graphicx}
\usepackage{float}
\graphicspath{ {img/} }
\udpschool{Escuela de Informática y Telecomunicaciones}

\begin{document}
\maketitle
\tableofcontents
\listoffigures
\chapter{Introducción}
  En este laboratorio se comprobara el funcionamiento del protocolo STP (IEEE 802.1D),  esto se llevara a cabo gracias al uso del programa CISCO Packet Tracer, también se implementara el protocolo de  VLAN (IEEE 802.1Q), de esta manera podremos entender cómo funciona, qué  ventajas tiene, y como implementarlo de manera eficiente.
\chapter{Actividades}
	\section{Topología base con bucles}
	
	{\large \bf{Cuestionario: }}\\
	\begin{enumerate}
	    \item ¿Qué camino realizara un paquete que para llegar desde el switch
0 hasta el switch2?\\\\
    Como el protocolo STP no afectó al enlace entre estos switches, el paquete se envía de forma directa sin pasar por otro switch.
    
        \item ¿Qué camino realizara un paquete que para llegar desde el switch
2 hasta el switch1?\\\\
    Como el protocolo STP afectó al enlace entre estos switches dejando el puerto que va desde el switch 1 al switch 2 como bloqueado. El camino recorrido por el paquete será: Switch 2 - Switch 0 - Switch 1. 
	\end{enumerate}
	\section{Configuración de STP}
	
	{\large \bf{Cuestionario: }}\\
	\begin{enumerate}
	    \item ¿Qué camino realizará un paquete que para llegar desde el switch
2 hasta el switch0?\\\\
        Como el protocolo STP afectó al enlace entre estos switches dejando el puerto que va desde el switch 2 al switch 0 como bloqueado. El camino recorrido por el paquete será: Switch 2 - Switch 1 - Switch 0. 
        \item  ¿Qué camino realizará un paquete que para llegar desde el switch
1 hasta el switch0?\\\\
          Como el protocolo STP no afectó al enlace entre estos switches, el paquete se envía de forma directa sin pasar por otro switch.
	\end{enumerate}
	\section{Priorización STP}
	
	\section{Configuración VLAN}
	
	{\large \bf{Cuestionario: }}\\
	\begin{enumerate}
	    \item ¿Cuál es la diferencia del modo Access y el modo Trunk en un switch?\\\\
	         El modo Trunk se utiliza para configurar las conexiones de switch a switch, indicando las vlan que pueden pasar por ese enlace. Por otro lado el modo Access es para conexiones desde el switch a un host, para indicar la vlan que puede pasar por el canal (por lo general es la misma vlan a la que pertenece el host).
        \item  ¿Qué ocurre si conecto una puerta en modo Trunk a un PC?\\\\
            Llegarían al PC todos los paquetes provenientes de las VLAN permitidas en la configuración del modo Trunk del respectivo puerto.
        \item ¿Qué ocurre si conecto dos switches, uno en modo access y otro en modo trunk?\\\\
              Transmitirían entre ellos solo los paquetes pertenecientes a la VLAN 
  	      permitida en el Switch configurado en modo Access.
  	     \item  ¿Qué camino realizara un paquete que para llegar desde el switch 1 hasta el switch 0?\\\\
  	     
	\end{enumerate}
\chapter{Conclusión}
  	      En esta experiencia de laboratorio se logra comprender cómo crear correctamente un paquete con Scapy,
  	      entendiendo todos sus componentes y cómo se puede enviar al o a los destinatarios que nosotros deseemos, también
  	      logra dejar clara la diferencia entre los comandos “send()” y “sendp()”, la capa en la que trabajan y cuando usar cada
  	      uno,se capta la forma en como funciona un Switch y un Hub con respecto al envío de paquetes y se aprendió trabajar con
  	      Wireshark para encontrar los paquetes del tipo que queramos o provenientes de la IP que cada uno  escoja.
  	      
 %MODIFICAR BIBLIOGRAFÍA ENTERA !! APRENDAN LATEX ! gg wp
\begin{thebibliography}{x}
\bibitem{Scapy Documentation} \textsc{Scapy Documentation},
\textit{http://www.secdev.org/projects/scapy/files/scapydoc.pdf}
\bibitem{Cisco Osi} \textsc{Cisco},
\textit{ http://www.cisco.com/cpress/cc/td/cpress/fund/ith/ith01gb.htm#xtocid166844}
\end{thebibliography}
% -----------------------------------------------------------------------------------------------
\end{document}
